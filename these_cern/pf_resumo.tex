\vfill
\begin{center}
\section*{Resumo\label{Resumo}}
\end{center}
\addcontentsline{toc}{chapter}{Resumo}

A engenharia e suas ferramentas têm encontrado sua aplicação em diversos campos da ciência.
Na física de partículas, um dos ambientes que está no limiar da ciência atual, encontra-se
o maior acelerador de partículas já construído. O \acrshort{lhc}, localizado na fronteira 
entre a Suíça e a França permitirá aos cientistas desenvolver e validar os modelos teóricos 
previstos, como o \glslink{mp}{Modelo Padrão}.

A partícula denominada de bóson de Higgs teve sua existência comprovada pelos dois maiores
experimentos já construídos pelo homem. Os detectores \acrshort{atlas} e \acrshort{cms} observaram 
ocorrências desses eventos nas colisões que ocorreram nos anos de 2011 e 2012. O de bóson de Higgs 
é altamente instável, o que faz com que ele decaia rapidamente em outras partículas, 
como eléctrons e  fótons, de forma que é de extrema importância a detecção das mesmas no experimento. 
Uma das dificuldades do experimento deve-se ao fato de que o bóson é extremamente raro e diversas colisões são 
necessárias para se observar o evento.

No detector \acrshort{atlas}, o maior experimento instalado no \acrshort{lhc}, os feixes de partículas
são direcionados para colidirem no centro do detector. Grande parte do que é gerado
pela colisão não representa a física de interesse e deve ser eliminado da cadeia de processamento 
de eventos. Para realizar a filtragem online de eventos no detector \acrshort{atlas}, um sistema de trigger 
foi implementado com o objetivo de reduzir a taxa de eventos armazenados ao final de cada colisão e maximizar
a probabilidade de detecção das partículas de interesse para os físicos, como os eléctrons e fótons.

Dentro desse contexto de filtragem online de eventos, o presente trabalho realiza a continuação do projeto
\acrshort{ringer} no sistema de \acrshort{trigger} do \acrshort{atlas}. O projeto consiste de um algoritmo para 
a identificação de elétrons, utilizando a informação especialista do detetor que é, então, propagada para um 
método estatístico de discriminação, atualmente formado por Redes Neurais. De modo a avaliar novo algoritmo, os 
resultados de eficiência dos discriminadores no canal de identificação de elétrons foram 
comparados com o atual algoritmo usado pela colaboração. O algoritmo proposto superou o atual no trigger nas 
bases de dados de simulação de Monte Carlo utilizadas como estudo deste trabalho.

\paragraph*{}

\noindent Palavra-chave: ATLAS, Calorimetria, Sistema de Filtragem, Ringer, Redes Neurais.

\vfill

\clearpage

% Abstract
\vfill
\begin{center}
\section*{Abstract\label{Abstract}}
\end{center}
\addcontentsline{toc}{chapter}{Abstract}

The engineering and  it is tools have found their application in various fields of science.
In particle physics, one of the environments that are in the cutting edge of Science, is
the largest particle accelerator ever built. The \acrshort{lhc}, located on the border
between Switzerland and France, will allow scientists to develop and validate several theoretical models, such as 
\glslink{mp}{Standard Model}.

The particle called Higgs boson had its existence proven by the two largest
experiments ever built by man. Detectors \acrshort{atlas} and \acrshort{cms} observed
occurrences of these events in collisions that occurred between the years 2011 and 2012. The decay of the 
Higgs boson is highly unstable, which causes it to decay rapidly into other particles, such as electrons and
photons, so that it is extremely importance detection in the same experiment. The difficulty of the experiment 
is due to the fact that the Higgs is extremely rare and many collisions are required to observe the event.

The detector \acrshort{atlas}, the largest experiment installed in the \acrshort{lhc}, the particle beam
is directed to collide in the center of the detector. Much of this product generated from
the collision is not the physics of interest and should be eliminated from the event processing chain. 
To perform the online event filtering for the \acrshort{atlas} detector, a trigger system has been implemented
 in order to reduce the rate of events stored at the end of each collision and maximize
the probability of detecting particles of interest for the physical, such as photons and electrons.

Within this online filtering context of events, this paper makes the continuation of the project
\acrshort{ringer} in the \acrshort{trigger} system from \acrshort{atlas}. The design consists of an algorithm 
for the identification of electrons using the detector expert information that is then propagated to a statistical 
method of discrimination, currently formed by Neural Networks. In order to measure the new algorithm,
 the efficiency of the discriminating results in the electron channel identification were compared with the actual 
 algorithm used by the collaboration. The proposed algorithm outperformed the current trigger on the Monte Carlo
  simulation databases used in this work.

\paragraph*{}

\noindent Key-words: ATLAS, Calorimetry, Trigger, Ringer,  Neural Networks.

\vfill
\clearpage
