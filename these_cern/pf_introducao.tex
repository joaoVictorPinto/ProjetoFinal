\chapter{Introdução}
\label{cap:intro}
\glsresetall

Diversas aplicações na engenharia estão relacionadas em ambientes cuja a alta taxa
de eventos e a rara ocorrência são uma dificuldade a ser superada. Diversas técnicas
e ferramentas  disponíveis  atualmente podem ser utilizadas como solução para tais
problemas. Técnicas de Inteligência artificial, como \acrlongpl{rna} são utilizadas para 
selecionar eficientemente a informação relevante ao experimento em tempo real. 
Posteriormente, os eventos selecionados armazenados são analisados utilizando-se 
de técnicas mais refinadas a fim de se obter melhores resultados que aqueles realizados 
durante a etapa de seleção em tempo real.

Em aplicações onde utiliza-se uma quantidade grande de sensores para descrever
com precisão o fenômeno de interesse e cuja a alta taxa é relativamente alta. Novamente,
as técnicas de inteligência artificial, Processamento de Sinais Estocástico utilizadas para selecionar
eficientemente o evento e algoritmos de compactação de informação utilizados para
extrair as características  do sinal podem ser utilizadas para reduzir o fluxo e quantidade
de informação a ser analisada na aplicação

\section{Motivação}

O \gls{cern} é o maior centro de pesquisas em física de partículas do mundo, situado na  fronteira da Suíça com 
a França. O experimento de maior repercussão do \gls{cern} atualmente é o \gls{lhc}, um acelerador de partículas 
de 27~km de circunferência que irá atingir energia de colisões nunca antes obtidas experimentalmente. 

Neste experimento, de proporções de colaboração internacional, serão acelerados pacotes\footnote{Um aglomerado de partículas.}
de prótons em feixes com sentidos opostos onde ocorrerão colisões em quatro pontos. Nesses pontos, foram instalados 
detectores de partículas. O \gls{atlas}  é o maior detetor do mundo, construído por uma colaboração internacional de 
174 institutos e 38 países. Este detetor foi projetado de modo a atender os diversos requisitos solicitados pelos
estudiosos da area de forma a realizar o estudo geral da matéria. Com todos esses requisitos, o detetor possui aproximadamente 
140 milhões de canais de leitura e está dividido nos seguintes subdetectores:

\begin{itemize}
\item \gls{id}, ou Inner Detector, é responsável pela detecção da trajetória de partículas carregadas;
\item \gls{ecal}, cujo objetivo é realizar a absorção total da energia de partículas eletromagnéticas;
\item \gls{hcal}, que de maneira similar ao eletromagnético, realiza a absorção da energia de partículas hadrônicas;
\item Espectrômetro de Múons, camada mais externa do detetor, realiza  a identificação e determinação da trajetória de Múons.
\end{itemize}

Devido a física de interesse ser extremamente rara, são necessários diversos dias de colisão para se
observar o evento de interesse. O detector  \gls{atlas}  foi concebido com o objetivo de identificar
principalmente a partícula de Higgs, descoberta entre as colisões de 2011 e 2012. Em números,
o subproduto da colisão de pacotes de prótons no detetor gera uma taxa de 40MHz de informação.
Se todos esses eventos produzidos fossem armazenados, o detetor produziria 60TB de informação
por segundo, tornando impossível seu armazenamento físico para as condições do projeto. Como 
solução, foi proposto um sistema de filtragem online de eventos.

O sistema de trigger do  \gls{atlas} é dividido em três níveis em sequência com o objetivo de reduzir
a taxa de armazenamento e maximizar a eficiência na detecção. As características de cada nível são
descritas a seguir:


\begin{itemize}
\item O \glslink{l1}{Primeiro Nível de filtragem} realiza a filtragem com
\gls{fpga}\footnote{Circuitos integrados digitais e programáveis.}, 
utilizando resolução reduzida das células do detector e ou o detector de múons
com um tempo fixo de 2,5~$\mu$s, reduzindo a taxa de eventos para
75~kHz. Ele também é responsável pela identificação de regiões no detector onde
há informação relevante, referidas como \gls{roi}. Somente a
informação contida nessa região é propagada para o segundo nível, de forma a
minimizar o fluxo de dados no sistema.

\item O \glslink{l2}{Segundo Nível de Trigger ou Fast High Level Trigger} 
acessa a \gls{roi} enviada pela aceitação do nível anterior e utiliza
a resolução total do detetor para executar a reconstrução e extração
de características do calorímetro e do  \gls{id}. O tempo de latência deste nível é de aproximadamente
10~ms, e deve reduzir a taxa de eventos para 2~kHz. É implementado em linguagens de alto
nível como C++ e python, o que permite rodar algoritmos mais sofisticados, desde que esses atendam
os requisitos de tempo requeridos por este nível.

\item O \glslink{ef}{Terceiro Nível ou Precision High Level Trigger}, além de avaliar com maior acurácia aqueles que foram
selecionados pelo segundo nível. Ele possui um tempo de  latência de 10~s, reduzindo a 
taxa de eventos para 100~Hz. Neste nível algoritmos de calibração, reconstrução e algoritmos de track, que possuem 
maior custo computacional, são executados utilizando total acesso a todos os canais do detetor para uma decisão mais acurada.

\end{itemize}

Os eventos aceitos pelo \gls{hlt} são armazenados para serem analisados a posteriori pelos físicos
responsáveis por cada canal de interesse. Nesta etapa, algoritmos ainda mais rebuscados são utilizados
para reconstruir com precisão máxima a informação da partícula no detector. Um dos maiores desafios
do sistema de filtragem \textit{online} de eventos é maximizar a eficiência na detecção das partículas estudadas
e minimizar a taxa de armazenamento de partículas desinteressantes para o processo. 

Um dos canais de interesse do experimento, o Canal \acrshort{eg}, deseja identificar elétrons \glslink{eg}{(e$^-$)}, 
pósitrons \glslink{eg}(e$^+$) ou fótons \glslink{eg}{($\gamma$)}, partículas de componentes eletromagnética. 
Muitos dos decaimentos do bóson de Higgs serão nessas partículas, de forma que esse canal é de fundamental 
importância para o experimento. Partículas onde componentes hadrônica são predominantes, mascaram a 
assinatura das partículas desejadas pelo Canal \acrshort{eg}, fazendo com que a tarefa da identificação dessas 
partículas não seja trivial.

Assim, um dos objetivos dos algoritmos de extração e de hipótese no sistema de filtragem nesse canal é
discriminar elétrons e fótons de jatos. Visando melhorar o desempenho do sistema de filtragem do detetor, 
foi proposto um algoritmo de extração de características aneladas no calorímetro no segundo nível de filtragem. 
Em conjunto com esse algoritmo foi implementado um sistema baseado em \gls{rna} como discriminador 
para filtragens de partículas nesse canal.



\section{Objetivos} 
O propósito deste trabalho é apresentar um sistema de filtragem com alta eficiência de detecção de partículas
no canal \acrshort{eg}, mais especificamente dos elétrons decorrentes do decaimento $H\rightarrow 2Z\rightarrow 4e$. Além do desenvolvimento 
do discriminador, toda a parte de infraestrutura de código precisou ser implementada dentro do ambiente utilizado pela colaboração. 
Em decorrência da grande quantidade de dados avaliados e da necessidade de maior poder computacional para realizar
o treinamento e avaliação dos discriminadores, em paralelo, foi desenvolvido um \textit{framework}\footnote{uma abstração que une códigos 
comuns entre vários projetos de software provendo uma funcionalidade genérica} de treinamento 
de redes neurais em C++ e python capaz de utilizar o poder da computação paralela internacional utilizada pelo \gls{cern}.

\newpage

\section{Organização do documento}{{{



O Capítulo~\ref{cap:cern} é dedicado ao \gls{cern} e ao contexto de colaboração internacional
no qual o projeto foi desenvolvido, dando a base necessária para o entendimento
do experimento \gls{atlas}. Foi realizada uma introdução sobre a Física de
Partículas, uma apresentação ao acelerador \gls{lhc} e os componentes do detector \gls{atlas}.

Por sua vez, o Capítulo~\ref{cap:trigger} é destinado ao sistema de filtragem. Nele será detalhado todo o atual
sistema de \textit{trigger} utilizado no \gls{atlas}, No final deste capítulo também será detalhado os principais
algoritmos de discriminação e extração de características utilizados no sistema de \textit{trigger} de alto nível.

Dando continuidade, o Capítulo~\ref{cap:metodologia} será detalhado o algoritmo \textit{Neural Ringer}, proposta 
deste trabalho, todo o processo de treinamento e  o sistema implementado em conjunto com a computação paralela do \gls{cern}.
O Capítulo~\ref{cap:resultados} será apresentado os resultados e comparações com os algoritmos de referência e por fim 
o Capítulo~\ref{cap:conclusao}  será apresentada as conclusões e trabalhos futuros.

}}}







